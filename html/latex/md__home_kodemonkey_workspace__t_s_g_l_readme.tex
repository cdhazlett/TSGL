Thread Safe Graphics Library

You can generate Doxygen locally using 'make docs', or view the \href{http://calvin-cs.github.io/TSGL/html/index.html}{\tt T\-S\-G\-L A\-P\-I here}. 

 \subsection*{Description }

T\-S\-G\-L is a thread-\/safe graphics library perfect for drawing graphics. You can do a wide variety of things with T\-S\-G\-L, including\-: image manipulation and rendering (.bmp, .jpeg, and .png image formats supported), 2\-D polygon drawing (rectangles, circles, triangles, etc.), text rendering, animations with keyboard and/or mouse events, and much more. All drawing and rendering is done with threads and in parallel. This library is currently supported on Windows, Mac O\-S, and Linux. 3\-D graphics are currently not supported by this library.

If you would like T\-S\-G\-L in your local git repository, use the following command\-:

git clone \href{https://github.com/Calvin-CS/TSGL.git}{\tt https\-://github.\-com/\-Calvin-\/\-C\-S/\-T\-S\-G\-L.\-git}

Otherwise, click the \char`\"{}\-Download zip\char`\"{} button to download a zipped up version. 

 \subsection*{Goals }

The main goal of this library is to provide a thread-\/safe graphics library for 2\-D graphics. Other goals include\-: Helping beginning programming students learn about the complex process of parallelization by giving them hands-\/on tools to use in order to learn about parallelization without having them to worry about the problems associated with parallelization such as race conditions, mutexes, and more. It also helps educators teach programming students about parallelization through simple visualizations. 

 \subsection*{Installation }

To install T\-S\-G\-L, please see our \href{https://github.com/Calvin-CS/TSGL/wiki/Installing-TSGL}{\tt installing T\-S\-G\-L page}. 

 \subsection*{Tutorials }

Want to learn how to utilize the T\-S\-G\-L library? Check out the \href{https://github.com/Calvin-CS/TSGL/wiki}{\tt wiki} pages for some tutorials! 

 \subsection*{docs and docs-\/wiki }

docs and docs-\/wiki are both submodules located inside of the T\-S\-G\-L root directory. docs contains the documentation for T\-S\-G\-L classes and docs-\/wiki contains the wiki pages. See the R\-E\-A\-D\-M\-E\-M\-I\-S\-C.\-txt file for information on how to initialize and update these submodules in your local git repository. 