So, you\textquotesingle{}ve downloaded and installed T\+S\+G\+L. Congrats! Now what? Well, there\textquotesingle{}s a {\itshape lot} you can do with T\+S\+G\+L. For now, let\textquotesingle{}s start by making a simple Hello World program!

Start off by creating a file name it hello.\+cpp.

We\textquotesingle{}ll be writing in C++, so let\textquotesingle{}s place our \#include and using directives\+:


\begin{DoxyCode}
\textcolor{preprocessor}{#include <tsgl.h>}
\textcolor{keyword}{using namespace }\hyperlink{namespacetsgl}{tsgl};
\end{DoxyCode}


\hyperlink{tsgl_8h_source}{tsgl.\+h} contains \#include directives for all of the necessary header files needed in order to use the T\+S\+G\+L library. This includes vital class header files such as those for the Canvas class and the Timer class. T\+S\+G\+L also has its own namespace which must be used when using the T\+S\+G\+L library.

Moving forward, let\textquotesingle{}s add some code that will create and initialize a Canvas. A Canvas is essentially a screen that draws and displays whatever it is that you want to draw and display. There\textquotesingle{}s a special kind of Canvas, the Cartesian\+Canvas, that we will look at later on.

For now, let\textquotesingle{}s focus on the normal Canvas\+:


\begin{DoxyCode}
\textcolor{keywordtype}{int} main() \{
  Canvas c(0, 0, 600, 350, \textcolor{stringliteral}{"Hello World!"});
  c.start();
  c.wait();
\}
\end{DoxyCode}


This is essentially the skeleton code for any T\+S\+G\+L program. Let\textquotesingle{}s break it down. In the main method, a Canvas object is created centered at (0, 0) (on your monitor) with a width of 600 and a height of 350. It has a title, \char`\"{}\+Hello World!\char`\"{}.

The c.\+start() and c.\+wait() statements tell the Canvas to start drawing and then wait to close once it has completed drawing.

In between the c.\+start() and c.\+wait() statements is where the magic happens. We will place a new statement which will draw a line to the Canvas\+:


\begin{DoxyCode}
\textcolor{preprocessor}{#include <tsgl.h>}
\textcolor{keyword}{using namespace }\hyperlink{namespacetsgl}{tsgl};

\textcolor{keywordtype}{int} main() \{
  \hyperlink{classtsgl_1_1_canvas}{Canvas} c(0, 0, 600, 350, \textcolor{stringliteral}{"Hello World!"});
  c.start();
  \textcolor{comment}{//'H'}
  c.drawLine(100, 50, 100, 300);
  c.drawLine(100, 150, 165, 150);
  c.drawLine(165, 50, 165, 300);

  \textcolor{comment}{//'E'}
  c.drawLine(200, 50, 280, 50);
  c.drawLine(200, 50, 200, 300);
  c.drawLine(200, 150, 280, 150);
  c.drawLine(200, 300, 280, 300);

  \textcolor{comment}{//The two 'L's}
  c.drawLine(300, 50, 300, 300);
  c.drawLine(300, 300, 320, 300);
  c.drawLine(350, 50, 350, 300);
  c.drawLine(350, 300, 370, 300);

  \textcolor{comment}{//'O'}
  c.drawCircle(480, 180, 100, 64, BLACK, \textcolor{keyword}{false});
  c.wait();
\}
\end{DoxyCode}


We will examine the draw\+Line() and draw\+Circle() methods in the \mbox{[}\mbox{[}Using Shapes\mbox{]}\mbox{]} tutorial. For now, just know that draw\+Line() draws a line to the Canvas and draw\+Circle() draws a circle to the Canvas.

Our code is now complete! Compile the code, and then once it has compiled correctly, run it. A window should pop up with \char`\"{}\+Hello\char`\"{} written on it. To close the window, press the E\+S\+C key or click the \textquotesingle{}x\textquotesingle{} in one of the corners.

Please note that the c.\+start() and c.\+wait() methods {\itshape M\+U\+S\+T} be used at least once when working with the Canvas class. Bad things will happen if you do not have these methods and decide to draw something on a Canvas.

Also, the special Canvas, Cartesian\+Canvas, handles lines and circles in much the same way as the standard Canvas does (but on a Cartesian coordinate system instead).

You\textquotesingle{}ve just made your first T\+S\+G\+L project, congratulations!

Next up, in \mbox{[}\mbox{[}Using Shapes\mbox{]}\mbox{]}, we take a look at how to draw shapes onto a Canvas. 