So, you\textquotesingle{}ve downloaded and installed T\+S\+G\+L. Congrats! Now what? Well, there\textquotesingle{}s a {\itshape lot} you can do with T\+S\+G\+L. For now, let\textquotesingle{}s start by making a simple Hello World program!

Start off by creating a file name it hello.\+cpp.

We\textquotesingle{}ll be writing in C++, so let\textquotesingle{}s place our \#include and using directives\+:


\begin{DoxyCode}
\textcolor{preprocessor}{#include <tsgl.h>}
\textcolor{keyword}{using namespace }\hyperlink{namespacetsgl}{tsgl};
\end{DoxyCode}


\hyperlink{tsgl_8h_source}{tsgl.\+h} contains \#include directives for all of the necessary header files needed in order to use the T\+S\+G\+L library. This includes vital class header files such as those for the Canvas class and the Timer class. T\+S\+G\+L also has its own namespace which must be used when using the T\+S\+G\+L library.

Moving forward, let\textquotesingle{}s add some code that will create and initialize a Canvas. A Canvas is essentially a screen that draws and displays whatever it is that you want to draw and display. There\textquotesingle{}s a special kind of Canvas, the Cartesian\+Canvas, that we will look at later on.

For now, let\textquotesingle{}s focus on the normal Canvas\+:


\begin{DoxyCode}
\textcolor{keywordtype}{int} main() \{
  Canvas c(0, 0, 200, 300, \textcolor{stringliteral}{"Hello World!"});
  c.start();
  c.wait();
\}
\end{DoxyCode}


This is essentially the skeleton code for any T\+S\+G\+L program. Let\textquotesingle{}s break it down. In the main method, a Canvas object is created centered at (0, 0) (on your monitor) with a width of 200 and a height of 300. It has a title, \char`\"{}\+Hello World!\char`\"{}.

The c.\+start() and c.\+wait() statements tell the Canvas to start drawing and then wait to close once it has completed drawing.

In between the c.\+start() and c.\+wait() statements is where the magic happens. We will place a new statement which will draw text to the Canvas\+:


\begin{DoxyCode}
\textcolor{preprocessor}{#include <tsgl.h>}
\textcolor{keyword}{using namespace }\hyperlink{namespacetsgl}{tsgl};

\textcolor{keywordtype}{int} main() \{
  \hyperlink{classtsgl_1_1_canvas}{Canvas} c(0, 0, 200, 300, \textcolor{stringliteral}{"Hello World!"});
  c.start();
  c.setFont(\textcolor{stringliteral}{"assets/freefont/FreeMono.ttf"});
  c.drawText(\textcolor{stringliteral}{"Hello World!"}, 50, 150, 20, BLACK);
  c.wait();
\}
\end{DoxyCode}


c.\+set\+Font() sets the font of the text to draw. ({\itshape N\+O\+T\+E\+:} If you ever want to draw text, make sure you set the font first! See the freetype folder located in the assets folder of the T\+S\+G\+L source code for available fonts.).

c.\+draw\+Text() draws text to the Canvas with the top left part of the text starting at (50, 150) and then it is drawn 20 pixels wide in black (B\+L\+A\+C\+K is a T\+S\+G\+L color constant. There are multiple color constants available such as R\+E\+D, B\+L\+U\+E, O\+R\+A\+N\+G\+E, and so on. We will take a look at these later.).

Our code is now complete! Compile the code, and then once it has compiled correctly, run it. A window should pop up with \char`\"{}\+Hello World!\char`\"{} located in the center. To close the window, press the E\+S\+C key or click the \textquotesingle{}x\textquotesingle{} in one of the corners.

Please note that the c.\+start() and c.\+wait() methods {\itshape M\+U\+S\+T} be used at least once when working with the Canvas class. Bad things will happen if you do not have these methods and decide to draw something on a Canvas.

Also, the special Canvas, Cartesian\+Canvas, handles text in much the same way as the standard Canvas does (but on a Cartesian coordinate system instead).

You\textquotesingle{}ve just made your first T\+S\+G\+L project, congratulations!

Next up, in \mbox{[}\mbox{[}Using Shapes\mbox{]}\mbox{]}, we take a look at how to draw shapes onto a Canvas. 