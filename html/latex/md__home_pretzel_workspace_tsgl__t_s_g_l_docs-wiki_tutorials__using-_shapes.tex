T\+S\+G\+L has a wide assortment of shapes. Rectangles, triangles, concave and convex polygons are just some of the shapes that you can draw. Each shape has its own class that inherits from a generalized Shape class. To draw a given shape, the Canvas class has draw methods that will create and draw a corresponding shape on a Canvas object as well as shapes that do not have classes of their own (circles, for example).

This means that you do not have to create a shape object and then draw it on the Canvas; the draw methods built into the Canvas class take care of that for you and draw the shape accordingly.

Let\textquotesingle{}s look at some examples.

$\ast$$\ast$$\ast$\+Linux/\+Mac user\+:$\ast$$\ast$$\ast$ Create a new folder and name it \char`\"{}\+Tutorial2\char`\"{}. Create a file inside of that folder and call it \char`\"{}shapes.\+cpp\char`\"{}. Copy over the generic Makefile from the T\+S\+G\+L-\/master folder and change the \char`\"{}\+T\+A\+R\+G\+E\+T\char`\"{} line so that it now says \char`\"{}shapes\char`\"{} instead of \char`\"{}program\char`\"{}.

$\ast$$\ast$$\ast$\+Windows users\+:$\ast$$\ast$$\ast$ Create a Solution folder and name it \char`\"{}\+Tutorial2\char`\"{}. Add a Visual Studio project to that folder and name it \char`\"{}\+Shape\char`\"{}. Go to the Property Manager and right click on that project. Add the test\+Properties Property sheet (located in the T\+S\+G\+L-\/master folder). Go back to the Solution Explorer and open up Tutorial2. Add a .cpp to Shape and name it \char`\"{}shapes.\+cpp\char`\"{}.

$\ast$$\ast$$\ast$\+All three platforms\+:$\ast$$\ast$$\ast$ Follow the steps in the \mbox{[}\mbox{[}Building Programs\mbox{]}\mbox{]} page on how to compile and run the program (Linux/\+Mac users, this is a single-\/file program).


\begin{DoxyCode}
\textcolor{preprocessor}{#include <tsgl.h>}
\textcolor{keyword}{using namespace }\hyperlink{namespacetsgl}{tsgl};

\textcolor{keywordtype}{int} main() \{
  \hyperlink{classtsgl_1_1_canvas}{Canvas} c(0, 0, 500, 600, \textcolor{stringliteral}{"Shapes!"});
  c.start();
  c.drawCircle(250, 300, 50, 32);
  c.wait();
\}
\end{DoxyCode}


Compile and run it. A window should appear with a circle drawn in the center.

The draw\+Circle() method takes these parameters\+:


\begin{DoxyItemize}
\item x-\/coordinate for the center of the circle ({\ttfamily 250}).
\item y-\/coordinate for the center of the circle ({\ttfamily 300}).
\item The radius of the circle ({\ttfamily 50}).
\item The resolution of the circle (32 or 64; 64 makes it look nicer and smoother) ({\ttfamily 32}).
\item The color of the circle (optional parameter; set to {\ttfamily B\+L\+A\+C\+K} by default).
\item A boolean that determines whether the circle is filled or not (optional parameter; set to {\ttfamily true} by default).
\end{DoxyItemize}

The draw\+Circle() function lets you specify the two optional arguments; the color to be used (default {\ttfamily B\+L\+A\+C\+K}) and a boolean indicating whether the circle should be filled (default {\ttfamily true}) like so\+:


\begin{DoxyCode}
c.drawCircle(250, 300, 50, 32, RED, \textcolor{keyword}{false});  \textcolor{comment}{//Color = RED, filled = false}
\end{DoxyCode}


Rectangles are drawn in a similar fashion\+:


\begin{DoxyCode}
c.drawRectangle(50, 100, 100, 200);
\end{DoxyCode}


draw\+Rectangle() takes in these parameters\+:


\begin{DoxyItemize}
\item x-\/coordinate for the top left corner of the rectangle ({\ttfamily 50}).
\item y-\/coordinate for the top left corner of the rectangle ({\ttfamily 100}).
\item x-\/coordinate for the bottom right corner of the rectangle ({\ttfamily 100}).
\item y-\/coordinate for the bottom right corner of the rectangle ({\ttfamily 200}).
\item The color of the rectangle (optional parameter; set to {\ttfamily B\+L\+A\+C\+K} by default).
\item A boolean that determines whether the rectangle is filled or not (optional parameter; set to {\ttfamily true} by default).
\end{DoxyItemize}

How about a triangle? Well\+:


\begin{DoxyCode}
c.drawTriangle(150, 100, 250, 200, 150, 300);
\end{DoxyCode}


draw\+Triangle() takes in these parameters\+:


\begin{DoxyItemize}
\item x-\/coordinate for the first point of the triangle ({\ttfamily 150}).
\item y-\/coordinate for the first point of the triangle ({\ttfamily 100}).
\item x-\/coordinate for the second point of the triangle ({\ttfamily 250}).
\item y-\/coordinate for the second point of the triangle ({\ttfamily 200}).
\item x-\/coordinate for the third point of the triangle ({\ttfamily 150}).
\item y-\/coordinate for the third point of the triangle ({\ttfamily 300}).
\item The color of the triangle (optional parameter; set to {\ttfamily B\+L\+A\+C\+K} by default).
\item A boolean that determines whether the triangle is filled or not (optional parameter; set to {\ttfamily true} by default).
\end{DoxyItemize}

In essence, whenever you draw a shape, the first few parameters are for the x and y-\/coordinates of the points of the shape and the last two are optional and are for the color and whether or not it should be filled.

Putting all of this code together\+:


\begin{DoxyCode}
\textcolor{preprocessor}{#include <tsgl.h>}
\textcolor{keyword}{using namespace }\hyperlink{namespacetsgl}{tsgl};

\textcolor{keywordtype}{int} main() \{
  \hyperlink{classtsgl_1_1_canvas}{Canvas} c(0, 0, 500, 600, \textcolor{stringliteral}{"Shapes!"});
  c.start();
  c.drawCircle(250, 300, 50, 32);  \textcolor{comment}{//Circle}
  c.drawRectangle(50, 100, 100, 200);  \textcolor{comment}{//Rectangle}
  c.drawTriangle(150, 100, 250, 200, 150, 300);  \textcolor{comment}{//Triangle}
  c.wait();
\}
\end{DoxyCode}


Compile and run it. A filled black circle, triangle, and rectangle should appear on a gray screen.

You can also draw regular lines\+:


\begin{DoxyCode}
c.drawLine(10, 20, 30, 40, PURPLE);
\end{DoxyCode}


draw\+Line() takes in these parameters\+:


\begin{DoxyItemize}
\item x-\/coordinate of the first point of the line ({\ttfamily 10}).
\item y-\/coordinate of the first point of the line ({\ttfamily 20}).
\item x-\/coordinate of the second point of the line ({\ttfamily 30}).
\item y-\/coordinate of the second point of the line ({\ttfamily 40}).
\item The color of the line ({\ttfamily P\+U\+R\+P\+L\+E}) (optional parameter; set to {\ttfamily B\+L\+A\+C\+K} by default.).
\end{DoxyItemize}

Check out the documentation for the \href{http://calvin-cs.github.io/TSGL/html/_canvas_8h_source.html}{\tt Canvas} class in the \mbox{[}\mbox{[}T\+S\+G\+L A\+P\+I\mbox{]}\mbox{]} page to learn more about its various shape drawing methods. You can also look at the various classes for the shapes to learn more about how the shapes are actually drawn onto the Canvas.

As we have said, there is a special Canvas called the Cartesian\+Canvas. That draws shapes in exactly the same way as the normal Canvas does (but on a Cartesian coordinate system).

We will take a closer look at the Cartesian\+Canvas in the \char`\"{}\+Using Functions\char`\"{} tutorial.

That concludes this tutorial!

In the next tutorial, \mbox{[}\mbox{[}Using Text\mbox{]}\mbox{]}, you get to learn about text! 