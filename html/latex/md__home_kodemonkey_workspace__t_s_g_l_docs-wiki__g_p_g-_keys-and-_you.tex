

 \subsection*{$\vert$ G\-P\-G K\-E\-Y\-S F\-O\-R S\-I\-G\-N\-I\-N\-G D\-E\-B\-I\-A\-N P\-A\-C\-K\-A\-G\-E\-S (Last updated\-: 08/09/16) $\vert$ }

$\ast$$\ast$$\ast$\-N\-O\-T\-E\-:$\ast$$\ast$$\ast$ This document attempts to guide you in the process of creating a G\-P\-G key. If there's something that we have missed, or perhaps have misleading information, please edit this page accordingly!

Before you can begin the process of creating a Debian package, you {\itshape N\-E\-E\-D} to have a G\-P\-G key. This is so that you can sign Debian packages and upload them to our P\-P\-A (see \mbox{[}\mbox{[}P\-P\-A and our T\-S\-G\-L team\mbox{]}\mbox{]}).

This is a security measure; Launchpad has you sign Debian packages with a G\-P\-G key.

Think of this as a stamp of approval that says, \char`\"{}this package does not contain malware\char`\"{}.

Also, if the package is tampered with while you are uploading it, the G\-P\-G key will reflect this. This will result in the package being rejected, since it was tampered with.

A G\-P\-G key has two parts\-: a public key that you can distribute freely to others, and a private key that you should N\-E\-V\-E\-R distribute.

In order to generate those parts, you need to use the {\ttfamily gpg} program via a terminal. (There's a G\-U\-I version that you can use called {\ttfamily seahorse}, but that process has never been tested).

Open up a terminal, and let's get started.

1). Type\-:

``` \begin{DoxyVerb}sudo apt-get install haveged
\end{DoxyVerb}


```

This daemon will help generate random bytes, used in the process of making a G\-P\-G key.

2). Type\-:

```

gpg --gen-\/key

```

You should now have the {\ttfamily gpg} program up and running. Something like this will show up\-:

``` \begin{DoxyVerb}Please select what kind of key you want:
(1) DSA and Elgamal (default)
(2) DSA (sign only)
(5) RSA (sign only)

Your selection? 
\end{DoxyVerb}


```

Type {\ttfamily 1}, then hit {\ttfamily E\-N\-T\-E\-R}.

3). Next prompt\-:

``` \begin{DoxyVerb}DSA keypair will have 1024 bits.
ELG-E keys may be between 1024 and 4096 bits long.
What keysize do you want? (2048)
\end{DoxyVerb}


```

Type {\ttfamily 2048}, and hit {\ttfamily E\-N\-T\-E\-R}.

4). Type {\ttfamily 0} for the next prompt, and hit {\ttfamily E\-N\-T\-E\-R}.

5). Type {\ttfamily yes} for the next prompt, and hit {\ttfamily E\-N\-T\-E\-R}.

6). Use your name and email for the {\ttfamily Real Name} and {\ttfamily Email} portion.

7). You will now be asked for a passphrase. Make this something easy for you to remember, but hard for others to guess. Keep this passphrase in mind, because you'll need it whenever you sign Debian packages with this new key.

8). Your key will now be generated.

Now that your key is generated, it's time to send it to the Ubuntu keyserver.

First, copy the {\ttfamily $<$key-\/id$>$} part of the text that was generated after your key was generated\-:

``` \begin{DoxyVerb}gpg: key <key-id> marked as ultimately trusted.
public and secret key created and signed. 

(more stuff here...)
\end{DoxyVerb}


```

Once you have the {\ttfamily $<$key-\/id$>$} copied, type\-:

``` \begin{DoxyVerb}gpg --send-keys --keyserver keyserver.ubuntu.com <key-id>
\end{DoxyVerb}


```

(Replacing {\ttfamily $<$key-\/id$>$} with the key id that you copied).

After waiting for 10 minutes or so, type\-:

``` \begin{DoxyVerb}gpg --list-keys
\end{DoxyVerb}


```

You should see something like this\-:

``` \begin{DoxyVerb}pub   #####/******** 2016-06-20
uid                  Real Name <Email@Email.com>
sub   #####/******** 2016-06-20
\end{DoxyVerb}


```

The {\ttfamily $\ast$$\ast$$\ast$$\ast$$\ast$$\ast$$\ast$$\ast$$\ast$} in the {\ttfamily sub} line is what you will use as your {\ttfamily $<$key-\/id$>$} when signing packages. You can also use the {\ttfamily $\ast$$\ast$$\ast$$\ast$$\ast$$\ast$$\ast$$\ast$$\ast$} in the {\ttfamily pub} line, but {\itshape P\-L\-E\-A\-S\-E} be consistent!

To get the G\-P\-G key fingerprint (which is what you'll need if you haven't seen our \mbox{[}\mbox{[}P\-P\-A and our T\-S\-G\-L team\mbox{]}\mbox{]} page!), type\-:

``` \begin{DoxyVerb}gpg --fingerprint
\end{DoxyVerb}


```

You should see something like this\-:

``` \begin{DoxyVerb}pub   #####/******** 2016-06-02
      Key fingerprint = **** **** **** **** ****  **** **** **** **** ****
uid                  Real Name <Email@Email.com>
sub   #####/******** 2016-06-02
\end{DoxyVerb}


{\ttfamily  The}Key fingerprint = $\ast$$\ast$$\ast$$\ast$ $\ast$$\ast$$\ast$$\ast$ $\ast$$\ast$$\ast$$\ast$ $\ast$$\ast$$\ast$$\ast$ $\ast$$\ast$$\ast$$\ast$ $\ast$$\ast$$\ast$$\ast$ $\ast$$\ast$$\ast$$\ast$ $\ast$$\ast$$\ast$$\ast$ $\ast$$\ast$$\ast$$\ast$ $\ast$$\ast$$\ast$$\ast$ ``` is your fingerprint.

You should now be ready to sign Debian packages!

If you haven't already, please see the \mbox{[}\mbox{[}P\-P\-A and our T\-S\-G\-L team\mbox{]}\mbox{]} page.

You may also want to take a look at the \mbox{[}\mbox{[}Saving G\-P\-G keys\mbox{]}\mbox{]} page, if you haven't already.

Otherwise, feel free to continue with the \mbox{[}\mbox{[}Creating New Deb Packages\mbox{]}\mbox{]} page. 

 \subsection*{$\vert$ M\-I\-S\-C $\vert$ }

For more information, please see this \href{http://ubuntuforums.org/showthread.php?t=680292}{\tt website}.

$\ast$$\ast$$\ast$\-P\-L\-E\-A\-S\-E$\ast$$\ast$$\ast$ update this document if you come across new links, or if there's missing/false information. 