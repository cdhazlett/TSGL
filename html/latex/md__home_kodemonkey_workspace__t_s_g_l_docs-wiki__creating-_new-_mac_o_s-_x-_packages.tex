$\ast$$\ast$$\ast$\-W\-O\-R\-K I\-N P\-R\-O\-G\-R\-E\-S\-S!!!"$\ast$$\ast$$\ast$ 

 \subsection*{$\vert$ C\-R\-E\-A\-T\-I\-N\-G A N\-E\-W M\-A\-C O\-S X I\-N\-S\-T\-A\-L\-L\-E\-R P\-A\-C\-K\-A\-G\-E (Last updated\-: 08/11/16) $\vert$ }

Creating a new {\ttfamily .pkg} Installer for Mac O\-S X isn't as difficult as creating an R\-P\-M or Debian package.

You do have to be on a Mac, however.

If you cannot procure a Mac with O\-S X 10.\-5 or greater (or a V\-M with O\-S X 10.\-5 or greater on it, {\itshape though this has not been tested}), $\ast$$\ast$$\ast$please stop reading and go get a Mac$\ast$$\ast$$\ast$.

Otherwise, please continue. 

 \subsection*{$\vert$ Packages $\vert$ }

When we first started to create the Mac O\-S X Installer package, we used a tool called {\ttfamily Packages}.

This made the process of creating a Mac O\-S X binary installer much easier and less tedious.

You have to download it from this \href{http://s.sudre.free.fr/Software/Packages/about.html}{\tt link}.

The download will be a {\ttfamily .dmg} file. Move it to your {\ttfamily Desktop}.

Now, mount it. (Double-\/click on it). You should see a Volume appear on your {\ttfamily Desktop} with the name, {\ttfamily Packages \#.\#.\#}. Double-\/click on it.

Now, double-\/click on the {\ttfamily Install Packages.\-pkg} file.

Follow the installation process, and you should have the {\ttfamily Packages} tool installed. 

 \subsection*{$\vert$ Creating the T\-S\-G\-L Installer $\vert$ }

After installing {\ttfamily Packages}, open up the application.

A screen should pop-\/up with the title, {\ttfamily New Project}.

This is where you choose what project you'd like to make.

We used the {\ttfamily Distribution} template, so go ahead and click on that option.

Click {\ttfamily Next}.

The project name will be {\ttfamily Install\-T\-S\-G\-L}, and the project directory will be {\ttfamily $\sim$/\-Desktop}.

A screen should appear showing you the project details.

In here, you can edit the settings for the project, how the Installer looks, and more.

On the left, you'll see a section named {\ttfamily P\-A\-C\-K\-A\-G\-E\-S}.

At the moment, there's only one package in it\-: {\ttfamily Install\-T\-S\-G\-L}.

We're going to rename it to just {\ttfamily T\-S\-G\-L}.

Double-\/click on it, and when the text appears highlighted, type {\ttfamily T\-S\-G\-L} and hit {\ttfamily E\-N\-T\-E\-R}.

Now, take a look at the {\ttfamily Settings} page.

This contains the settings that you can edit for this package.

If you take a look at the text box next to the {\ttfamily Identifier\-:} label, you can see that the Identifier for the package still has {\ttfamily Install\-T\-S\-G\-L} at the end.

We're going to change the {\ttfamily Install\-T\-S\-G\-L} part to say {\ttfamily T\-S\-G\-L}.

Click on the text box next to the {\ttfamily Identifier\-:} label, and delete the {\ttfamily Install\-T\-S\-G\-L} part. Replace it with {\ttfamily T\-S\-G\-L} and hit {\ttfamily E\-N\-T\-E\-R}.

We can now leave the {\ttfamily Settings} page as it is.

On the left, click on {\ttfamily Project}.

You'll be taken back to the {\ttfamily Settings} page for the Installer.

Leave the {\ttfamily Settings} page as is, and click on the {\ttfamily Presentation} tab.

In here, you'll be able to customize the text for the {\ttfamily Introduction}, {\ttfamily Read Me}, and {\ttfamily License} screens.

To do so, we must first create the text files that will hold the customized screen text.

Create three new text files\-: {\ttfamily intro.\-txt}, {\ttfamily R\-E\-A\-D\-M\-E.\-txt}, and {\ttfamily license.\-txt}.

We'll edit {\ttfamily intro.\-txt} first. 

 \subsection*{$\vert$ {\ttfamily intro.\-txt} $\vert$ }

This is the first thing that a user will see when they double-\/click on the Installer package.

Here's the introduction that we used\-:

``` Hello! Thank you for choosing to install T\-S\-G\-L on your Mac!

To start, please click \char`\"{}\-Continue\char`\"{}.

```

Copy and paste this into your {\ttfamily intro.\-txt} file and make any changes.

Save and close it when you are satisfied. 

 \subsection*{$\vert$ {\ttfamily R\-E\-A\-D\-M\-E.\-txt} $\vert$ }

This is the next thing that a user will see when they continue with the installation process.

Essentially, we specified what dependencies would be installed and with which package manager.

``` \begin{DoxyVerb}The following dependencies will be installed on your Mac:

    - X11 libs
    - Homebrew
    - Freetype (through Homebrew)
    - glfw3 (through Homebrew)
    - glew (through Homebrew)
    - doxygen (through Homebrew)
    - gcc5 (through Homebrew)

By clicking "Continue", you agree to having these dependencies installed. 
\end{DoxyVerb}


```

Copy the above text and paste it into your {\ttfamily R\-E\-A\-D\-M\-E.\-txt} file.

Make any changes as you see fit, then save and close the file. 

 \subsection*{$\vert$ {\ttfamily license.\-txt} $\vert$ }

This is the last thing a user will see before selecting a destination, choosing installation type, and installing T\-S\-G\-L on their Mac.

We took the opportunity to use the license associated with T\-S\-G\-L (G\-P\-Lv3).

This file is actually inside of the T\-S\-G\-L source code folder.

Simply open up the T\-S\-G\-L source code folder (the clone or downloaded zip file), open up {\ttfamily license}, and copy the text inside of it. Paste that text inside of {\ttfamily license.\-txt}.

Save and close the file. 

 \subsection*{$\vert$ Almost there $\vert$ }

Now that we have those three text files made, it's time to integrate them into the Installer.

Back in {\ttfamily Packages}, make sure that you are in the {\ttfamily Project} details (click on {\ttfamily Project} on the left side of the window), and on the {\ttfamily Presentation} page (click on the {\ttfamily Presentation} tab).

In the drop down box on the right, make sure that it says {\ttfamily Introduction}.

Then, click the big plus ({\ttfamily +}) button underneath the {\ttfamily Custom Introduction} table.

One of the table slots inside of the {\ttfamily Custom Introduction} table should be filled in and highlighted.

Click on the section that has the {\ttfamily -\/} in it, and click {\ttfamily Choose} from the drop down menu.

Double-\/click on the {\ttfamily intro.\-txt} file that we made earlier.

Now the text inside of the {\ttfamily Introduction} Installer page should change to the text inside of {\ttfamily intro.\-txt}.

Do the same thing for the {\ttfamily R\-E\-A\-D\-M\-E.\-txt} and {\ttfamily license.\-txt} files.

Once you have done that, you are almost ready to create the {\ttfamily .pkg} Installer! 

 \subsection*{$\vert$ Installer plugins $\vert$ }

Our Installer can have two shell scripts which are executed before and after T\-S\-G\-L is installed.

The {\ttfamily preinstall} and {\ttfamily postinstall} scripts, respectively.

We will go into more detail later on these scripts, but for now just know that they are needed for our Installer.

However, they are not the {\itshape O\-N\-L\-Y} things that are needed.

As we developed the {\ttfamily .pkg} Installer, we focused primarily on making it self-\/contained.

We wanted a user to simply mount our {\ttfamily .dmg} file, double-\/click on the {\ttfamily .pkg} file, and have T\-S\-G\-L installed on their Mac.

We didn't want them to execute any programs beforehand; we wanted everything to be done by the {\ttfamily .pkg} Installer.

This proved to be most difficult with just the {\ttfamily preinstall} and {\ttfamily postinstall} scripts.

Which is why, another resource was needed\-: an Installer plugin.

An Installer plugin is essentially another pane that you can add to the installation process.

What's important about this resource?

You can execute commands based off of user input.

Even shell scripts.

This was {\itshape V\-E\-R\-Y} useful for us when we had to install {\ttfamily Homebrew} as a dependency, as well as a solution to many problems, for the reasons outlined in the next section. 

 \subsection*{$\vert$ The trouble with {\ttfamily preinstall} and {\ttfamily postinstall} $\vert$ }

As mentioned above, the {\ttfamily preinstall} and {\ttfamily postinstall} scripts are shell scripts which can be executed before and after T\-S\-G\-L is installed, respectively.

They are perfect for small tasks, like making symlinks or installing some dependencies.

However, they are run as {\ttfamily root}.

This doesn't seem like such a big deal, but consider this scenario\-:

1). The {\ttfamily preinstall} script is executed when the user clicks {\ttfamily Install} during the T\-S\-G\-L installation process.

2). The script finds that a user doesn't have {\ttfamily Homebrew} installed.

3). It attempts to install {\ttfamily Homebrew}.

4). Because the script was run as {\ttfamily root}, {\ttfamily Homebrew} refuses to install.

5). No dependencies are installed, and the installation process shows that T\-S\-G\-L was installed correctly (even though it wasn't).

See https\-://github.com/\-Homebrew/brew/blob/master/share/doc/homebrew/\-F\-A\-Q.\-md \char`\"{}the Homebrew F\-A\-Q page\char`\"{} for more information on why using {\ttfamily sudo} and {\ttfamily brew} is a bad thing.

This was a recurring problem, with many workarounds attempted.

One of them involved creating a separate executable that the user could run {\itshape before} the {\ttfamily .pkg} Installer, but that would mean that the Installer would no longer be self-\/contained. Strike 1.

You can execute the scripts as normal user by changing a setting inside of the {\ttfamily .pkg} file {\itshape after} it's been created. That would be a workaround.

However, that brings another problem to the table.

When we have to install {\ttfamily X11} libs, we use the {\ttfamily curl} command in order to download the {\ttfamily x\-Quartz} {\ttfamily .dmg} file which contains a {\ttfamily .pkg} file so that they can be installed.

We then open up that file and tell the user to follow the installation process.

But...two Installers {\itshape cannot} be running at the same time.

If they are, one will wait for the other to finish before beginning.

This {\itshape always} happened with the installation of the {\ttfamily X11} libs.

The installation would wait until {\itshape after} the installation of T\-S\-G\-L was completed, then move onto installing {\ttfamily X11} libs. T\-S\-G\-L needs those libs in order to compile, and because they were not installed {\itshape before} T\-S\-G\-L, this would mean that T\-S\-G\-L would fail to compile and would {\itshape N\-O\-T} be installed correctly.

A workaround for this would be to use the {\ttfamily installer} command and use it to install the {\ttfamily .pkg} file via a Terminal.

But...the execution of the {\ttfamily installer} command could cause the same issue (two Installer running at the same time).

Even if it were to work, this workaround brings with it a new problem.

For some odd reason, the commands executed by the {\ttfamily preinstall} script are asynchronous.

What this means is that when one command is executed, the next command doesn't wait until the first has terminated. It executes, and the successive commands do the same.

This is really bad in terms of installing dependencies.

To give an example, consider {\ttfamily Homebrew} again.

{\ttfamily Homebrew} needs to be installed before {\ttfamily glew}, {\ttfamily glfw3}, {\ttfamily doxygen}, {\ttfamily freetype}, and {\ttfamily gcc5}.

If a user doesn't have {\ttfamily Homebrew} installed, the script needs to install {\ttfamily Homebrew}.

As a consequence of the commands being executed asynchronously, the other dependencies will also be installed {\itshape regardless of the {\ttfamily Homebrew} install finishing or not}.

This would lead to another failed install. Strike 2.

There are other problems, but we won't go into them in detail.

We attempted numerous workarounds, ranging from having the {\ttfamily preinstall} script execute other install scripts, to using {\ttfamily sleep} and {\ttfamily wait} commands.

None of them provided a solid, foolproof binary Installer. Strike 3.

Given all of these reasons, we ultimately decided to go with Installer plugins.

They provided a solid, self-\/contained, foolproof Installer.

We used one to install {\ttfamily Homebrew}, and {\ttfamily X11} libs.

We also utilized the {\ttfamily sleep} command in order to avoid having multiple dependencies installed at once (e.\-g. {\ttfamily X\-Code Command Line Tools} and dependencies installed by {\ttfamily Homebrew}). 

 \subsection*{$\vert$ Back to Installer plugins $\vert$ }



 \subsection*{$\vert$ Building the Installer $\vert$ }



 \subsection*{$\vert$ Wrapping it up in a .dmg file $\vert$ }

Now that you have the {\ttfamily .pkg} file created, it's time to wrap it up in a {\ttfamily .dmg} file.

Open up the {\ttfamily Disk Utility} application (which should be located in a folder called {\ttfamily Utilities}).

This is where we'll be creating the .dmg file.

That concludes this page!



 \subsection*{$\vert$ M\-I\-S\-C $\vert$ }

For more help with the Packages tool, please see this website\-: \href{http://s.sudre.free.fr/Software/documentation/Packages/en/index.html}{\tt http\-://s.\-sudre.\-free.\-fr/\-Software/documentation/\-Packages/en/index.\-html} 