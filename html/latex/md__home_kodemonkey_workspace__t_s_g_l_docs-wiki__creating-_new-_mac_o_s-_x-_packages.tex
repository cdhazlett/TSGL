

 \subsection*{$\vert$ C\-R\-E\-A\-T\-I\-N\-G A N\-E\-W M\-A\-C O\-S X I\-N\-S\-T\-A\-L\-L\-E\-R P\-A\-C\-K\-A\-G\-E (Last updated\-: 07/22/16) $\vert$ }

$\ast$$\ast$$\ast$\-W\-O\-R\-K I\-N P\-R\-O\-G\-R\-E\-S\-S!!!"$\ast$$\ast$$\ast$
\begin{DoxyItemize}
\item Need to finalize this page.
\end{DoxyItemize}

Creating a new .pkg Installer for Mac O\-S X isn't as difficult as R\-P\-M or Debian package creation.

You do have to be on a Mac, however.

If you cannot procure a Mac with O\-S X 10.\-5 or greater (or a V\-M with O\-S X 10.\-5 or greater on it, {\itshape though this has not been tested}), $\ast$$\ast$$\ast$please stop reading and go get a Mac$\ast$$\ast$$\ast$.

Otherwise, please continue. 

 \subsection*{$\vert$ Packages $\vert$ }

When we first started to create the Mac O\-S X Installer package, we used a tool called {\ttfamily Packages}.

This made the process of creating a Mac O\-S X binary installer much easier and less tedious.

You have to download it from this \href{http://s.sudre.free.fr/Software/Packages/about.html}{\tt link}.

The download will be a {\ttfamily .dmg} file. Move it to your {\ttfamily Desktop}.

Now, mount it. (Double-\/click on it). You should see a Volume appear on your {\ttfamily Desktop} with the name, {\ttfamily Packages \#.\#.\#}. Double-\/click on it.

Now, double-\/click on the {\ttfamily Install Packages.\-pkg} file.

Follow the installation process, and you should have the {\ttfamily Packages} tool installed. 

 \subsection*{$\vert$ Creating the T\-S\-G\-L Installer $\vert$ }

After installing {\ttfamily Packages}, open up the application.

A screen should pop-\/up with the title, {\ttfamily New Project}.

This is where you choose what project you'd like to make.

We used the {\ttfamily Distribution} template, so go ahead and click on that option.

Click {\ttfamily Next}.

The project name will be {\ttfamily Install\-T\-S\-G\-L}, and the project directory will be {\ttfamily $\sim$/\-Desktop}.

A screen should appear showing you the project details.

In here, you can edit the settings for the project, how the Installer looks, and more.

On the left, you'll see a section named {\ttfamily P\-A\-C\-K\-A\-G\-E\-S}.

At the moment, there's only one package in it\-: {\ttfamily Install\-T\-S\-G\-L}.

We're going to rename it to just {\ttfamily T\-S\-G\-L}.

Double-\/click on it, and when the text appears highlighted, type {\ttfamily T\-S\-G\-L} and hit {\ttfamily E\-N\-T\-E\-R}.

Now, take a look at the {\ttfamily Settings} page.

This contains the settings that you can edit for this package.

If you take a look at the text box next to the {\ttfamily Identifier\-:} label, you can see that the Identifier for the package still has {\ttfamily Install\-T\-S\-G\-L} at the end.

We're going to change the {\ttfamily Install\-T\-S\-G\-L} part to say {\ttfamily T\-S\-G\-L}.

Click on the text box next to the {\ttfamily Identifier\-:} label, and delete the {\ttfamily Install\-T\-S\-G\-L} part. Replace it with {\ttfamily T\-S\-G\-L} and hit {\ttfamily E\-N\-T\-E\-R}.

We can now leave the {\ttfamily Settings} page as it is.

On the left, click on {\ttfamily Project}.

You'll be taken back to the {\ttfamily Settings} page for the Installer.

Leave the {\ttfamily Settings} page as is, and click on the {\ttfamily Presentation} tab.

In here, you'll be able to customize the text for the {\ttfamily Introduction}, {\ttfamily Read Me}, and {\ttfamily License} screens.

To do so, we must first create the text files that will hold the customized screen text.

Create three new text files\-: {\ttfamily intro.\-txt}, {\ttfamily R\-E\-A\-D\-M\-E.\-txt}, and {\ttfamily license.\-txt}.

We'll edit {\ttfamily intro.\-txt} first. 

 \subsection*{$\vert$ {\ttfamily intro.\-txt} $\vert$ }

This is the first thing that a user will see when they double-\/click on the Installer package.

Here's the introduction that we used\-:

``` Hello! Thank you for choosing to install T\-S\-G\-L on your Mac.

The installation process should take about 22-\/45 minutes.

```

Copy and paste this into your {\ttfamily intro.\-txt} file and make any changes.

Save and close it when you are satisfied. 

 \subsection*{$\vert$ {\ttfamily R\-E\-A\-D\-M\-E.\-txt} $\vert$ }

This is the next thing that a user will see when they continue with the installation process.

Essentially, we specified what dependencies would be installed and with which package manager.

``` \begin{DoxyVerb}The following dependencies will be installed on your Mac using Homebrew:

    - doxygen
    - gcc49 <--- (Takes the longest to install with Homebrew)
    - freetype
    - glew
    - glfw3

By continuing, you agree to having these components on your Mac for TSGL. 
\end{DoxyVerb}


```

Copy the above text and paste it into your {\ttfamily R\-E\-A\-D\-M\-E.\-txt} file.

Make any changes as you see fit, then save and close the file. 

 \subsection*{$\vert$ {\ttfamily license.\-txt} $\vert$ }

This is the last thing a user will see before selecting a destination, choosing installation type, and installing T\-S\-G\-L on their Mac.

We took the opportunity to use the license associated with T\-S\-G\-L (G\-P\-Lv3).

This file is actually inside of the T\-S\-G\-L source code folder.

Simply open up the T\-S\-G\-L source code folder (the clone or downloaded zip file), open up {\ttfamily license}, and copy the text inside of it. Paste that text inside of {\ttfamily license.\-txt}.

Save and close the file. 

 \subsection*{$\vert$ Homestretch $\vert$ }

Now that we have those three text files made, it's time to integrate them into the Installer.

Back in {\ttfamily Packages}, make sure that you are in the {\ttfamily Project} details (click on {\ttfamily Project} on the left side of the window), and on the {\ttfamily Presentation} page (click on the {\ttfamily Presentation} tab).

In the drop down box on the right, make sure that it says {\ttfamily Introduction}.

Then, click the big plus ({\ttfamily +}) button underneath the {\ttfamily Custom Introduction} table.

One of the table slots inside of the {\ttfamily Custom Introduction} table should be filled in and highlighted.

Click on the section that has the {\ttfamily -\/} in it, and click {\ttfamily Choose} from the drop down menu.

Double-\/click on the {\ttfamily intro.\-txt} file that we made earlier.

Now the text inside of the {\ttfamily Introduction} Installer page should change to the text inside of {\ttfamily intro.\-txt}.



 \subsection*{$\vert$ Building the Installer $\vert$ }



 \subsection*{$\vert$ Wrapping it up in a .dmg file $\vert$ }

That concludes this page!



 \subsection*{$\vert$ M\-I\-S\-C $\vert$ }

For more help with the Packages tool, please see this website\-: \href{http://s.sudre.free.fr/Software/documentation/Packages/en/index.html}{\tt http\-://s.\-sudre.\-free.\-fr/\-Software/documentation/\-Packages/en/index.\-html} 